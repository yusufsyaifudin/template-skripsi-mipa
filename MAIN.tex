% -----------------------------------------------------------------------
% Template Skripsi untuk MIPA
% 
% @author Yusuf Syaifudin
% @created 29/02/2016
% 
% -----------------------------------------------------------------------

\documentclass[ugmskripsi]{ugmskripsi}

% ------------------------------------------------------------------------------
% Berisi tambahan package dan konfigurasi untuk masing-masing package.
% Ada baiknya, setiap konfigurasi diletakkan tepat dibawah 
% setelah package dilakukan import (usepackage) agar tidak membingungkan.
% Serta disarankan untuk menambah kegunaan package tersebut agar tidak lupa.
% ------------------------------------------------------------------------------


% font tambahan
\usepackage{textcomp}

% digunakan untuk membuat flowchart
\usepackage{tikz}
\usetikzlibrary{shapes,arrows, fit, positioning}

\usepackage{float}
\usepackage{booktabs}
\usepackage{pbox}
\usepackage{multirow}
\usepackage[normalem]{ulem}
\useunder{\uline}{\ul}{}

% Untuk hyperlink dan otomatis membuat bookmark
\usepackage{hyperref}

% break tanda /, - dan spasi ke baris baru jika sudah tidak muat
\def\UrlBreaks{\do\/\do-\do\ }

% font url dibuat miring dan dg jenis font ttfamily
\renewcommand{\UrlFont}{\small\ttfamily\itshape}

\usepackage{csquotes}
\usepackage{framed}
\usepackage{enumitem}

% untuk input kode baik dari file atau bukan
\usepackage{listings} 

% ----------------------------------------------------------------------------
% Contoh dari file
% ----------------------------------------------------------------------------
% \begin{figure}[H]
%   \lstinputlisting[language=python, firstline=38, lastline=59]{code/linkwalker.py}
%   \caption{Mendapatkan daftar tautan berita pada kompas.com}
%   \label{grab daftar berita kompas}
% \end{figure}
% ----------------------------------------------------------------------------
% 
% ----------------------------------------------------------------------------
% Contoh
% ----------------------------------------------------------------------------
% \begin{figure}
% 	\begin{lstlisting}[language=sql]
% 		update train_data_statement set data = replace(data, '“', '"');
% 		update train_data_statement set data = replace(data, '”', '"');
		
% 		update test_data_statement set data = replace(data, '“', '"');
% 		update test_data_statement set data = replace(data, '”', '"');
% 	\end{lstlisting}
% 	\caption{\textit{Query} SQL untuk melakukan perubahan karakter pada data}
% 	\label{kueri SQL untuk melakukan perubahan karakter pada data}
% \end{figure} 
% ------------------------------------------------------------------------------

\usepackage{color}
\usepackage{amsmath}
\usepackage{courier}
\usepackage[scaled=.75]{beramono}

%-----------------------------------------------------------------
% Setting syntax hightlighting
%-----------------------------------------------------------------
\lstset{frame=tb,
  language=Python,
  aboveskip=2mm,
  belowskip=1mm,
  showstringspaces=false,
  columns=flexible,
  basicstyle  = \fontfamily{pcr}\fontsize{8pt}{8pt}\selectfont,
  numbersep=8pt,
  numbers=left,
  numberstyle=\tiny\color{gray},
  keywordstyle=\color{blue},
  commentstyle=\color{dkgreen},
  stringstyle=\color{mauve},
  breaklines=true,
  breakatwhitespace=true,
  tabsize=4
}

% Untuk menghilangkan titik-titik pada daftar isi

\usepackage[titles]{tocloft}
\renewcommand{\cftdot}{}

% Untuk membuat multi kolom
\usepackage{etoolbox,refcount}
\usepackage{multicol}

% Konfigurasi multi kolom
% bikin multi kolom
\newcounter{countitems}
\newcounter{nextitemizecount}
\newcommand{\setupcountitems}{%
  \stepcounter{nextitemizecount}%
  \setcounter{countitems}{0}%
  \preto\item{\stepcounter{countitems}}%
}
\makeatletter
\newcommand{\computecountitems}{%
  \edef\@currentlabel{\number\c@countitems}%
  \label{countitems@\number\numexpr\value{nextitemizecount}-1\relax}%
}
\newcommand{\nextitemizecount}{%
  \getrefnumber{countitems@\number\c@nextitemizecount}%
}
\newcommand{\previtemizecount}{%
  \getrefnumber{countitems@\number\numexpr\value{nextitemizecount}-1\relax}%
}
\makeatother    
\newenvironment{AutoMultiColItemize}{%
\ifnumcomp{\nextitemizecount}{>}{2}{\begin{multicols}{2}}{}%
\setupcountitems\begin{itemize}}%
{\end{itemize}%
\unskip\computecountitems\ifnumcomp{\previtemizecount}{>}{2}{\end{multicols}}{}}
%end bikin multi kolom

% ------------------------------------------------------------------------------
% Contoh sintaks:
% ------------------------------------------------------------------------------
% \begin{itemize}
%   \item \textit{Reporting verb} yang hadir sebelum entitas pada kutipan langsung:
%   \begin{AutoMultiColItemize}
% 	  \item tutur
% 	  \item kata
% 	  \item ujar
%   \end{AutoMultiColItemize}

%   \item \textit{Reporting verb} yang hadir setelah entitas pada kutipan langsung:
%   \begin{AutoMultiColItemize}
% 	  \item mengatakan
% 	  \item menjawab
%   \end{AutoMultiColItemize}
% \end{itemize}
% ------------------------------------------------------------------------------


% Setting list agar spasi antar list tidak terlalu banyak
\setlist{  
  listparindent=\parindent,
  parsep=0pt
}

% Agar tetap Justify tapi kata tidak dipisah sesuka hati (not hypenation but justified)
\tolerance=1
\emergencystretch=\maxdimen
\hyphenpenalty=10000
\hbadness=10000
\hyphenchar\font=-1
\sloppy

% Agar referensi di paragraf auto menggunakan kata "Gambar" atau "Tabel"
% Contoh: Pada Gambar 3.2 di atas tampak ...
\makeatletter
\renewcommand*{\p@figure}{Gambar~}
\renewcommand*{\p@table}{Tabel~}
\makeatother

% Memudahkan pewarnaan pada cell tabel
\usepackage{xcolor}
\usepackage{colortbl}

% Konfigurasi variable seperti judul dan lain sebagainya
% Judul Bahasa Indonesia, ditulis kapital semua
\titleind{IDENTIFIKASI KALIMAT KUTIPAN DARI TEKS BERITA \textit{ONLINE} BERBAHASA INDONESIA DENGAN METODE BERBASIS ATURAN}

% Judul Bahasa Inggris, ditulis kapital semua
\titleeng{QUOTATIONS IDENTIFICATION FROM INDONESIAN ONLINE NEWS USING RULE-BASED METHOD}

% Nama Lengkap, ditulis huruf pertama setiap kata kapital
\fullname{Yusuf Syaifudin}

% NIM Lengkap
\idnum{11/313303/PA/13672}

% Tanggal Ujian
\examdate{16 Februari 2016}

% Gelar S1/S2
\shortdegree{S2}

% Gelar Lengkap, ditulis huruf pertama setiap kata kapital
\gelar{Sarjana Komputer}

% Tahun
\yearsubmit{2016}

% Program Studi, ditulis huruf pertama setiap kata kapital
\program{Ilmu Komputer}

% Kepala Program Studi
\headprogram{Azhari SN, Dr., MT}

% Departemen / Jurusan, ditulis huruf pertama setiap kata kapital
\dept{Ilmu Komputer dan Elektronika}

% Pembimbing pertama, ditulis huruf pertama setiap kata kapital
\firstsupervisor{Arif Nurwidyantoro, S.Kom., M.Cs}

% NIP Pembimbing
\nipfirstsupervisor{123-456-789-1111}

% Penguji Pertama, ditulis huruf pertama setiap kata kapital
\firstexaminer{Mhd. Reza M.I Pulungan, M.Sc., Dr.-Ing}

% Penguji Kedua, ditulis huruf pertama setiap kata kapital
\secondexaminer{Sigit Priyanta, S.Si., M.Kom}

% Penguji Ketiga (untuk Tesis), ditulis huruf pertama setiap kata kapital
\thirdexaminer{Afiahyati, Ph.D}

% Tesis / Skripsi, ditulis huruf pertama setiap kata kapital
\reporttype{Tesis}





\begin{document}

%-----------------------------------------------------------------
% Disini awal masukan untuk muka skripsi
%-----------------------------------------------------------------

% Cover
\cover

% Halaman judul
\titlepageind 

% Halaman Persetujuan
\ifdefined\approvalproposalpage
\approvalproposalpage
\fi

% Halaman Pengesahan
\ifdefined\approvalpage
\approvalpage
\fi

% Halaman Pernyataan
\declarepage

% Halaman Persembahan
\acknowledment
\input{BAB0/1_PERSEMBAHAN}

%-----------------------------------------------------------------
% Disini akhir masukan untuk muka skripsi
%-----------------------------------------------------------------

% Motto
\input{BAB0/2_MOTTO}

% Prakata
\input{BAB0/3_PRAKATA}

%-----------------------------------------------------------------
% Daftar Isi
%-----------------------------------------------------------------
\newpage
\phantomsection
\addcontentsline{toc}{chapter}{\contentsname}
\tableofcontents
%-----------------------------------------------------------------
% Akhir Daftar Isi
%-----------------------------------------------------------------

%-----------------------------------------------------------------
% Daftar Tabel
%-----------------------------------------------------------------
\newpage
\phantomsection
\addcontentsline{toc}{chapter}{\listtablename}
\listoftables
%-----------------------------------------------------------------
% Akhir Daftar Tabel
%-----------------------------------------------------------------

%-----------------------------------------------------------------
% Daftar Gambar
%-----------------------------------------------------------------
\newpage
\phantomsection
\addcontentsline{toc}{chapter}{\listfigurename}
\listoffigures
%-----------------------------------------------------------------
% Akhir Daftar Gambar
%-----------------------------------------------------------------


%-----------------------------------------------------------------
%Disini awal masukan Intisari
%-----------------------------------------------------------------
\begin{abstractind}
	\input{BAB0/4_ABSTRAK_ID}
\end{abstractind}
%-----------------------------------------------------------------
%Disini akhir masukan Intisari
%-----------------------------------------------------------------

%-----------------------------------------------------------------
%Disini awal masukan untuk Abstract
%-----------------------------------------------------------------
\begin{abstracteng}
  \input{BAB0/5_ABSTRAK_EN}
\end{abstracteng}
%-----------------------------------------------------------------
%Disini akhir masukan Abstract
%-----------------------------------------------------------------


%-----------------------------------------------------------------
% Awal BAB 1
%-----------------------------------------------------------------
\chapter{PENDAHULUAN}
\label{PENDAHULUAN}

	\section{Latar Belakang}
	\label{pendahuluan latar belakang}
	\input{BAB1/1_LATAR_BELAKANG}

	\section{Rumusan Masalah}
	\label{pendahuluan rumusan masalah}
	\input{BAB1/2_RUMUSAN_MASALAH}

	\section{Batasan Masalah}
	\label{pendahuluan batasan masalah}
	\input{BAB1/3_BATASAN_MASALAH}

	\section{Tujuan Penelitian}
	\label{pendahuluan tujuan penelitian}
	\input{BAB1/4_TUJUAN_PENELITIAN}

	\section{Manfaat Penelitian}
	\label{pendahuluan manfaat penelitian}
	\input{BAB1/5_MANFAAT_PENELITIAN}

	\section{Sistematika Penulisan}
	\label{pendahuluan sistematika penulisan}
	\input{BAB1/6_SISTEMATIKA_PENULISAN}

%-----------------------------------------------------------------
% Akhir BAB 1
%-----------------------------------------------------------------


%-----------------------------------------------------------------
% Awal BAB 2
%-----------------------------------------------------------------
\chapter{TINJAUAN PUSTAKA}
\label{TINJAUAN PUSTAKA}
\input{BAB2/1_TINJAUAN_PUSTAKA}

%-----------------------------------------------------------------
% Akhir BAB 2
%-----------------------------------------------------------------


%-----------------------------------------------------------------
% Awal BAB 3
%-----------------------------------------------------------------
\chapter{DASAR TEORI}
\label{DASAR TEORI}

	\section{Representational State Transfer}
	\label{dasar teori rest}
	\input{BAB3/1_REST}

	\section{JavaScript Object Notation}
	\label{dasar teori json}

		\subsection{Definisi}
		\label{dasar teori definisi json}
		\input{BAB3/2_1_DEFINISI_JSON}

		\subsection{Contoh}
		\label{dasar teori contoh json}
		\input{BAB3/2_2_CONTOH_JSON}

%-----------------------------------------------------------------
% Akhir BAB 3
%-----------------------------------------------------------------


%-----------------------------------------------------------------
% Awal BAB 4
%-----------------------------------------------------------------
\chapter{ANALISIS DAN PERANCANGAN SISTEM}
\label{ANALISIS DAN PERANCANGAN SISTEM}

	\section{Deskripsi Umum Sistem}
	\label{rancangan deskripsi umum sistem}
	\input{BAB4/1_DESKRIPSI_UMUM}

	\section{Analisis Kebutuhan Sistem}
	\label{rancangan analisis kebutuhan sistem}
	\input{BAB4/2_ANALISIS_KEBUTUHAN_SISTEM}

	\section{Pembuatan Sistem}
	\label{rancangan pembuatan sistem}

		\subsection{Pembuatan Sistem Pengenalan Entitas Bernama}
		\label{rancangan pembuatan sistem pengenalan entitas bernama}
		\input{BAB4/3_2_SISTEM_PENGENALAN_ENTITAS_BERNAMA}

		\subsection{Pembuatan Sistem Ekstraksi Kalimat Pernyataan}
		\label{rancangan sistem ekstraksi kalimat pernyataan}
		\input{BAB4/3_3_SISTEM_EKSTRAKSI_KALIMAT_PERNYATAAN}

	\section{Rancangan Antarmuka}
	\label{rancangan antarmuka}

		\subsection{Deskripsi}
		\label{rancangan deskripsi antarmuka}
		\input{BAB4/4_1_DESKRIPSI_RANCANGAN_ANTARMUKA}

		\subsection{\textit{Wireframe}}
	    \label{rancangan wireframe antarmuka}
	    \input{BAB4/4_2_WIREFRAME_ANTARMUKA}

%-----------------------------------------------------------------
% Akhir BAB 4
%-----------------------------------------------------------------


%-----------------------------------------------------------------
% Awal BAB 5
%-----------------------------------------------------------------
\chapter{IMPLEMENTASI SISTEM}
\label{IMPLEMENTASI SISTEM}

	\section{Spesifikasi}
	\label{implementasi spesifikasi}
	\input{BAB5/1_SPESIFIKASI}

	\section{Implementasi Sistem Pengenalan Entitas Bernama}
	\label{implementasi sistem ner}
	\input{BAB5/2_1_IMPLEMENTASI_SISTEM_NER}

	\section{Implementasi Sistem Ekstraksi Kalimat Pernyataan}
	\label{implementasi sistem ekstraksi kalimat pernyataan}
	\input{BAB5/2_2_IMPLEMENTASI_SISTEM_EKTRAKSI_KALIMAT_PERNYATAAN}

%-----------------------------------------------------------------
% Akhir BAB 5
%-----------------------------------------------------------------



%-----------------------------------------------------------------
% Awal BAB 6
%-----------------------------------------------------------------
\chapter{PENGUJIAN DAN PEMBAHASAN SISTEM}
\label{PENGUJIAN DAN PEMBAHASAN SISTEM}
\input{BAB6/1_PENDAHULUAN}

	\section{Pengujian Sistem Pengenalan Entitas Bernama}
	\label{pengujian sistem ner}
	\input{BAB6/2_PENGUJIAN_SISTEM_NER}

	\section{Pengujian Sistem Ekstraksi Kalimat Pernyataan}
	\label{pengujian sistem ekstraksi kalimat pernyataan}
	\input{BAB6/3_PENGUJIAN_SISTEM_EKSTRAKSI_KALIMAT_PERNYATAAN}

%-----------------------------------------------------------------
% Akhir BAB 6
%-----------------------------------------------------------------


%-----------------------------------------------------------------
% Awal BAB 7
%-----------------------------------------------------------------
\chapter{PENUTUP}
\label{PENUTUP}

	\section{Kesimpulan}
	\label{penutup kesimpulan}
	\input{BAB7/1_KESIMPULAN}

	\section{Saran}
	\label{penutup saran}
	\input{BAB7/2_SARAN}

%-----------------------------------------------------------------
% Akhir BAB 7
%-----------------------------------------------------------------

%-----------------------------------------------------------------
% Awal Daftar Pustaka
%-----------------------------------------------------------------
\input{BAB8_DAFTAR_PUSTAKA/1_DAFTAR_PUSTAKA_MIPA}
%-----------------------------------------------------------------
% Akhir Daftar Pustaka
%-----------------------------------------------------------------


%-----------------------------------------------------------------
% Awal lampiran
%-----------------------------------------------------------------
\appendix

\chapter{BERKAS JSON UNTUK MODEL SISTEM PENGENALAN ENTITAS BERNAMA}
\label{BERKAS JSON UNTUK MODEL SISTEM PENGENALAN ENTITAS BERNAMA}
\input{BAB9_LAMPIRAN/1_BERKAS_MODEL_NER}

%-----------------------------------------------------------------
% Akhir lampiran
%-----------------------------------------------------------------

\end{document}